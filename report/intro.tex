\section*{Введение}
Наше исследование посвящено решению краевой задачи с помощью PINN (Physics-Informed Neural Networks)
Дифференциальные уравнения --- очень важная часть математики, это целый математический аппарат (Дифференциальные уравнения в частных производных ) широко применяемый при разработке моделей в  разных областях науки и техники.

К сожалению, явное решение этих уравнений в аналитическом виде оказывается возможным только в частных простых случаях.
Поэтому для анализа математических моделей, основанных на дифференциальных уравнениях, используются  приближенные численные методы.
\par В нашем исследовании мы используем метод PINN, он отличается от классических методов (Метод конечных разностей, метод конечных элементов, метод конечных объемов и пр.)
и обьединяет методы машинного обучения с фундаментальными физическими законами.

\medskip
\medskip

При этом для численных методов характерены: большой обьем вычислений, использование вычислительных систем. Поэтому с появлением нейропроцессоров и цифровых сигнальных процессоров метод становится особенно интересным, ведь может дать выигрыш в скорости выполнения, а также он может помочь в случаях, когда традиционные численные методы сталкиваются с трудностями. (Этот метод предлагает новый взгляд на решение сложных физических задач, особенно в случаях, когда традиционные численные методы сталкиваются с трудностями.)