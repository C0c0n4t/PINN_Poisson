\section{Физическое описание}
Уравнение Пуассона --- эллиптическое дифференциальное уравнение в частных производных.

\textbf{Где можно встретить данное уравнение:}

\subsection{ Электростатика: }

\begin{equation}
  \nabla^2 \varphi = -\frac{\rho}{\varepsilon_0},
\end{equation}

где
\begin{itemize}
  \setlength{\itemsep}{-10pt}
  \item $\varphi$ --- электростатический потенциал (В),\\
  \item $\rho$ --- объемная плотность заряда ($\text{Кл}/\text{м}^{3}$),\\
  \item $\varepsilon_0$ — диэлектрическая проницаемость вакуума ($\text{Ф}/\text{м}$),\\
  \item $\nabla = \left( \frac{\partial}{\partial x}, \frac{\partial}{\partial y}, \frac{\partial}{\partial z} \right)$ — оператор Набла.
\end{itemize}

\par Вывод уравнения из закона Гаусса $\mathrm{div}\,\mathbf{E} \equiv \nabla \cdot \mathbf{E} = \rho/ \varepsilon_0$ и определения статического потенциала $(\mathbf{E} = -\nabla \varphi)$
($\mathbf{E}$ --- напряженность электрического поля):

\begin{equation}
  \frac{\rho}{\varepsilon_0} = \nabla \cdot \mathbf{E} = \nabla \cdot (-\nabla \varphi) = - \nabla \cdot \nabla \varphi = - \nabla^2 \varphi.
\end{equation}

\par (Еще одно применение, но такое же по сути) Если мы имеем объемную сферически симметричную плотность гауссового распределения заряда $\rho(r)$:

\begin{equation}
  \rho(r) = \frac{Q}{\sigma^3(\sqrt{2\pi})^3} e^{-r^2/(2\sigma^2)},
\end{equation}

где
\begin{itemize}
  \setlength{\itemsep}{-10pt}
  \item $E$ --- (В), \\
  \item $Q$ --- общий заряд системы, \\
  \item $\sigma$ --- поверхностная плотность заряда,
\end{itemize}

то решение $\Phi(r)$ уравнения Пуассона:

\begin{equation}
  \nabla^2 \Phi = -\frac{\rho}{\varepsilon_0},
\end{equation}

дается в виде:

\begin{equation}
  \Phi(r) = \frac{1}{4 \pi \varepsilon_0} \frac{Q}{r}\,\mathrm{erf}\left(\frac{r}{\sqrt{2}\sigma}\right),
\end{equation}

где $\mathrm{erf}(x)$ — функция ошибок.

Это решение может быть проверено напрямую вычислением $\nabla^2 \Phi$.

\subsection{ Гравитационные силы: }
Расчет гравитационного потенциала  $\varphi$. Его градиент определяет напряженность гравитационного поля.
Потенциал $\varphi$, создаваемый точечной массой $m$, расположенной в начале координат, равен

\begin{equation}
  \varphi_m = -\frac{G m}{r},
\end{equation}

где
\begin{itemize}
  \setlength{\itemsep}{-10pt}
  \item $G$ --- гравитационная постоянная,\\
  \item $r$ --- расстояние от начала координат.\\
\end{itemize}

\par
На бесконечности потенциал такого вида обращается в ноль. В общем случае произвольного распределения массы, описываемого координатно-зависимой плотностью $\rho$ ($\text{кг}/{\text{м}^3}$), уравнение Пуассона записывается:

\begin{equation}
  \nabla^2 \varphi = 4 \pi G \rho.
\end{equation}
С точностью до замены $-1/\varepsilon_0 \rightarrow 4 \pi G$ и изменения смысла величины $\rho$ («плотность заряда»  → «плотность массы»), уравнение подобно соответствующему электростатическому уравнению. Правда, в случае гравитационных сил не бывает ситуации отталкивания, но на решении этот факт никак не сказывается.

Решение имеет вид:

\begin{equation}
  \varphi (x,y,z) = \int \text{G} \frac{\rho(\xi,\eta,\zeta)}{\sqrt{(x-\xi)^2+(y-\eta)^2+(z-\zeta)^2}} d\xi d\eta d\zeta.
\end{equation}

Рассмотрение уравнения Пуассона в остальных областях физики (в которых он встречается) может быть выполнено аналогично, только со специфическим для конкретной области смыслом входящих в него величин.