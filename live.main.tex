undefined
Уравнение Пуассона — эллиптическое дифференциальное уравнение в частных производных.

\textbf{Где можно встретить данное уравнение:}

\begin{enumerate}
  \item 
  \item 

\subsection{ Электростатика: }

\begin{equation}
  \nabla^2 \varphi = -\frac{\rho}{\varepsilon_0},
\end{equation}

где $\varphi$ --- электростатический потенциал (В),\\
$\rho$ --- объемная плотность заряда ($\frac{\text{Кл}}{\text{м}^{3}}$),\\
$\varepsilon_0$ — диэлектрическая проницаемость вакуума ($\frac{\text{Ф}}{\text{м}}$),\\
$\nabla = \left( \frac{\partial}{\partial x}, \frac{\partial}{\partial y}, \frac{\partial}{\partial z} \right)$ — оператор Набла.
\\
Вывод уравнения - из закона Гаусса $(\mathrm{div}\,\mathbf{E} \equiv \nabla \cdot \mathbf{E} = \rho/ \varepsilon_0)$ и определения статического потенциала $(\mathbf{E} = -\nabla \varphi)$:

\begin{equation}
  \frac{\rho}{\varepsilon_0} = \nabla \cdot \mathbf{E} = \nabla \cdot (-\nabla \varphi) = - \nabla \cdot \nabla \varphi = - \nabla^2 \varphi.
\end{equation}
где $E$ — электростатический потенциал (В)\\

2) (Еще одно применение, но такое же по сути) Если мы имеем объемную сферически симметричную плотность гауссового распределения заряда $\rho(r)$:

\begin{equation}
  \rho(r) = \frac{Q}{\sigma^3\sqrt{2\pi}^3} e^{-r^2/(2\sigma^2)},
\end{equation}

где $Q$ — общий заряд, то решение $\Phi(r)$ уравнения Пуассона:

\begin{equation}
  \nabla^2 \Phi = -\frac{\rho}{\varepsilon_0},
\end{equation}

дается:

\begin{equation}
  \Phi(r) = \frac{1}{4 \pi \varepsilon_0} \frac{Q}{r}\,\mathrm{erf}\left(\frac{r}{\sqrt{2}\sigma}\right),
\end{equation}

где $\mathrm{erf}(x)$ — функция ошибок.

Это решение может быть проверено напрямую вычислением $\nabla^2 \Phi$.

\end{enumerate}

\subsection{ Гравитационные силы: }
1) Расчет гравитационного потенциала  $\varphi$. Его градиент определяет напряженность гравитационного поля.

Потенциал $\varphi$, создаваемый точечной массой $m$, расположенной в начале координат, равен

\begin{equation}
  \varphi_m = -\frac{G m}{r},
\end{equation}
где $G$     — гравитационная постоянная,\\
$r$ — расстояние от начала координат.\\
\par
На бесконечности потенциал такого вида обращается в ноль. В общем случае произвольного распределения массы, описываемого координатно-зависимой \href{https://ru.wikipedia.org/wiki/\%D0\%9F\%D0\%BB\%D0\%BE\%D1\%82\%D0\%BD\%D0\%BE\%D1\%81\%D1\%82\%D1\%8C}{плотностью} $ρ$ (кг/м\textsuperscript{3}), уравнение Пуассона записывается:

\begin{equation}
  \nabla^2 \varphi = 4 \pi G \rho.
\end{equation}
С точностью до замены $-\frac{1}{\varepsilon_0} \rightarrow 4 \pi G$    и изменения смысла величины $\rho$ («плотность заряда»  → «плотность массы»), уравнение подобно соответствующему электростатическому уравнению. Правда, в случае гравитационных сил не бывает ситуации отталкивания, но на решении этот факт никак не сказывается.

Решение такой же вид, как и в электростатике:

\begin{equation}
  \varphi (x,y,z) = \int \text{G} \frac{\rho(\xi,\eta,\zeta)}{\sqrt{(x-\xi)^2+(y-\eta)^2+(z-\zeta)^2}} d\xi d\eta d\zeta.
\end{equation}

\medskip
\medskip
\medskip
\medskip
\medskip
\medskip
\medskip
Рассмотрение уравнения Пуассона в остальных областях физики (в которых он встречается) может быть выполнено аналогично, только со специфическим для конкретной области смыслом входящих в него величин.
undefined